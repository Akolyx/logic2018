\documentclass[10pt,a4paper,oneside]{article}
\usepackage[utf8]{inputenc}
\usepackage[english,russian]{babel}
\usepackage{amsmath}
\usepackage{amsthm}
\usepackage{amssymb}
\usepackage{enumerate}
\usepackage{stmaryrd}
\usepackage{enumitem}
\usepackage[left=2cm,right=2cm,top=2cm,bottom=2cm,bindingoffset=0cm]{geometry}
\usepackage{proof}
\newcommand{\gq}[1]{\texttt{«}#1\texttt{»}}
\newcommand{\ogq}[1]{\overline{\texttt{«}#1\texttt{»}}}
\begin{document}

\section*{Разбор домашнего задания №1}

Первые три задачи мы разберём по возможности близко к определениям, либо строя 
требуемые выводы, либо явно ссылаясь на соответствующие теоремы, гарантирующие их 
существование. Последние две задачи мы приведём на более высокоуровневом языке, 
не сильно менее строгом, но более выразительном и более далёком от базовых определений.

Очень полезной при решении задач оказывается теорема о дедукции. Причём, в обе стороны.
Т. е. $\alpha\vdash\beta \iff \vdash\alpha\rightarrow\beta$


\begin{enumerate}[label=(\alph*)]
\item $\vdash\alpha\&\beta\rightarrow\beta\&\alpha$
\newline
  Нужно было доказать, что конъюнкция удовлетворяет закону коммутативности. 
  Идея в том, что зная $\alpha\&\beta$ мы с помощью схем аксиом разбиваем её 
  на $\alpha$ и $\beta$. Потом собираем в другом порядке.

Следующий вывод покажет $\alpha\&\beta\vdash\beta\&\alpha$:

\begin{enumerate}[label=(\arabic*)]
\item $\alpha\&\beta$ (гипотеза)
\item $\alpha\&\beta\rightarrow\alpha$ (сх. акс. 4)
\item $\alpha\&\beta\rightarrow\beta$ (сх. акс. 5)
\item $\alpha$ (M. P. 2, 1)
\item $\beta$ (M. P. 3, 1)
\item $\beta\rightarrow\alpha\rightarrow\beta\&\alpha$ (сх. акс. 3)
\item $\alpha\rightarrow\beta\&\alpha$ (M. P. 6, 5)
\item $\beta\&\alpha$ (M. P. 7, 4)
\end{enumerate}

Далее же, применив теорему о дедукции, 
получаем требуемое $\vdash\alpha\&\beta\rightarrow\beta\&\alpha$
  
\item $\vdash\alpha \rightarrow \neg\neg \alpha$
\newline
  Для доказательства этой теоремы можно воспользоваться тем, что если 
  формула $\alpha$ верна, то благодаря первой схеме аксиом мы можем 
  вывести и $\beta\rightarrow\alpha$ для произвольной формулы $\beta$. 
  В частности, будет верна и формула $\neg\alpha\rightarrow\alpha$. 
  Формулы $\neg\alpha\rightarrow\neg\alpha$ (доказана в конспекте) и 
  $\neg\alpha\rightarrow\alpha$ приводят нас к противоречию. 
  Доказательство формул от противного даёт нам девятая схема аксиом. 

  Поскольку в предметном языке у нас нет возможности использовать ссылки на конспект, 
  нам придётся переписывать вывод 
  $\neg\alpha\rightarrow\neg\alpha$ явно.

  Итак, покажем $\alpha\vdash\neg\neg\alpha$. В этом выводе
  высказывания с 1 по 3 нужны, чтобы показать
  $\alpha\vdash\neg\alpha\rightarrow\alpha$. Высказывания с 4 по 8 
  скопированы из конспекта (из леммы 4.2, с 
  подстановкой $\alpha := \neg\alpha$) --- так мы показываем
  $\alpha\vdash\neg\alpha\rightarrow\neg\alpha$.
  И заключительная часть (высказывания с 9 по 11) собственно и 
  содержит само приведение к противоречию.

\begin{enumerate}[label=(\arabic*)]
\item $\alpha$ (гипотеза)
\item $\alpha\rightarrow(\neg\alpha\rightarrow\alpha)$ (сх. акс. 1)
\item $\neg\alpha\rightarrow\alpha$ (M. P. 2, 1)
\item $\neg\alpha\rightarrow\neg\alpha\rightarrow\neg\alpha$ (сх. акс. 1)
\item $(\neg\alpha \rightarrow \neg\alpha \rightarrow \neg\alpha) \rightarrow 
  (\neg\alpha \rightarrow ((\neg\alpha \rightarrow \neg\alpha) \rightarrow \neg\alpha)) \rightarrow
  (\neg\alpha \rightarrow \neg\alpha)$ (Сх. акс. 2)
\item $(\neg\alpha \rightarrow ((\neg\alpha \rightarrow \neg\alpha) \rightarrow \neg\alpha)) \rightarrow
  (\neg\alpha \rightarrow \neg\alpha)$ (M.P. 5, 4)
\item $(\neg\alpha \rightarrow ((\neg\alpha \rightarrow \neg\alpha) \rightarrow \neg\alpha))$ (Сх. акс. 1)
\item $\neg\alpha\rightarrow\neg\alpha$ (M.P. 7, 6)
\item $(\neg\alpha\rightarrow\alpha)\rightarrow(\neg\alpha\rightarrow\neg\alpha)\rightarrow\neg\neg\alpha$ (сх. акс. 9)
\item $(\neg\alpha\rightarrow\neg\alpha)\rightarrow\neg\neg\alpha$ (M. P. 9, 3)
\item $\neg\neg\alpha$ (M. P. 10, 8)
\end{enumerate}

  Теперь, воспользовавшись теоремой о дедукции, получим 
  $\vdash\alpha\rightarrow\neg\neg\alpha$

\item $\vdash\alpha\&(\beta\vee\gamma) \rightarrow (\alpha\&\beta)\vee(\alpha\&\gamma)$
\newline
  В этой задаче необходимо сконструировать более сложное выражение (т.е.
  $(\alpha\&\beta)\lor(\alpha\&\gamma)$) из имеющихся у нас более простых
  (т.е. $\alpha$ и $\beta\lor\gamma$). Единственным способом получить дизъюнкцию
  справа от импликации являются схемы аксиом 6 и 7. Обратите внимание, что в
  схеме аксиом 8 также участвует дизъюнкция, но она дает дизъюнкицю слева от
  импликации (что нам тоже пригодится). Чтобы дать интуицию почему дизъюнкция не
  может просто сменить свою позицию относительно импликации (т.е. будучи слева
  перейти вправо или наоборот), можно привести следующий пример, где дизъюнкция
  превращается в конъюнкцию при смене стороны импликации; сравните: <<если дождь
  или снег, то осадки>> против <<если не осадки, то не дождь и не снег>>.

  Сперва мы выведем $(\alpha\&\beta)\vee(\alpha\&\gamma)$ из
  предположений $\alpha$ и $\beta$ и, отдельно, из предположений $\alpha$ и $\gamma$.
  Затем перенесём $\beta$ и $\gamma$ вправо с помощью теоремы о
  дедукции и введём дизъюнкцию слева от ипмпликации, соединив эти два 
  предположения в одно с помощью схемы аксиом 8.

\begin{enumerate}[label=(\Alph*)]
\item Сперва получим $\alpha,\beta\vdash(\alpha\&\beta)\vee(\alpha\&\gamma)$
\newline

  Доказательство:
\begin{enumerate}[label=(\arabic*)]
\item $\alpha$ (гипотеза)
\item $\beta$ (гипотеза)
\item $\alpha\rightarrow\beta\rightarrow\alpha\&\beta$ (сх. акс. 3)
\item $\beta\rightarrow\alpha\&\beta$ (M. P. 3, 1)
\item $\alpha\&\beta$ (M. P. 4, 2)
\item $\alpha\&\beta\rightarrow(\alpha\&\beta)\vee(\alpha\&\gamma)$ (сх. акс. 6)
\item $(\alpha\&\beta)\vee(\alpha\&\gamma)$ (M. P. 6, 5)
\end{enumerate}  
И теперь по теореме о дедукции получим
$$\alpha\vdash\beta\rightarrow(\alpha\&\beta)\vee(\alpha\&\gamma)$$
Соответствующий вывод назовём выводом A.

\item Сперва получим $\alpha,\gamma\vdash(\alpha\&\beta)\vee(\alpha\&\gamma)$
\newline
  Доказательство аналогично:
\begin{enumerate}[label=(\arabic*)]
\item $\alpha$ (гипотеза)
\item $\gamma$ (гипотеза)
\item $\alpha\rightarrow\gamma\rightarrow\alpha\&\gamma$ (сх. акс. 3)
\item $\gamma\rightarrow\alpha\&\gamma$ (M. P. 3, 1)
\item $\alpha\&\gamma$ (M. P. 4, 2)
\item $\alpha\&\gamma\rightarrow(\alpha\&\beta)\vee(\alpha\&\gamma)$ (сх. акс. 7)
\item $(\alpha\&\beta)\vee(\alpha\&\gamma)$ (M. P. 6, 5)
\end{enumerate}
И теперь по теореме о дедукции получим
$$\alpha\vdash\gamma\rightarrow(\alpha\&\beta)\vee(\alpha\&\gamma)$$
Соответствующий вывод назовём выводом B.

\item Используя выводы A и B, а также схему аксиом №8, получим вывод, который выводит
уже почти то, что нам нужно: 
\newline
$\vdash\alpha\rightarrow\beta\vee\gamma\rightarrow(\alpha\&\beta)\vee(\alpha\&\gamma)$
\newline 
Доказательство:
\begin{enumerate}[label=(\arabic*)]
\item[(...)] Текст вывода A, без заключительного высказывания
\item $\beta\rightarrow(\alpha\&\beta)\vee(\alpha\&\gamma)$ (заключительное высказывание вывода A)
\item[(...)] Текст вывода B, без заключительного высказывания
\item $\gamma\rightarrow(\alpha\&\beta)\vee(\alpha\&\gamma)$ (заключительное высказывание вывода B)
\item $(\beta\rightarrow(\alpha\&\beta)\vee(\alpha\&\gamma))\rightarrow(\gamma\rightarrow(\alpha\&\beta)\vee(\alpha\&\gamma))\rightarrow(\beta\vee\gamma\rightarrow(\alpha\&\beta)\vee(\alpha\&\gamma))$ (сх. акс. 8)
\item $(\gamma\rightarrow(\alpha\&\beta)\vee(\alpha\&\gamma))\rightarrow(\beta\vee\gamma\rightarrow(\alpha\&\beta)\vee(\alpha\&\gamma))$ (M. P. 3, 1)
\item $\beta\vee\gamma\rightarrow(\alpha\&\beta)\vee(\alpha\&\gamma)$ (M. P. 4, 2)
\end{enumerate}

К полученному тексту применим теорему о дедукции и получим:
$$\vdash\alpha\rightarrow\beta\vee\gamma\rightarrow(\alpha\&\beta)\vee(\alpha\&\gamma)$$
Соответствующий вывод назовём выводом C.

\item Теперь приведём утверждение в вид, который требовался в условии. Рассмотрим предоложение
$\alpha\&(\beta\vee\gamma)$:
\begin{enumerate}[label=(\arabic*)]
\item $\alpha\&(\beta\vee\gamma)$ (гипотеза)
\item $\alpha\&(\beta\vee\gamma)\rightarrow\alpha$ (сх. акс. 4)
\item $\alpha$ (M. P. 2, 1)
\item $\alpha\&(\beta\vee\gamma)\rightarrow(\beta\vee\gamma)$ (сх. акс. 5)
\item $\beta\vee\gamma$ (M. P. 4, 1)
\item[(...)] Текст вывода C без последнего высказывания
\item $\alpha\rightarrow\beta\vee\gamma\rightarrow(\alpha\&\beta)\vee(\alpha\&\gamma)$ (Последнее высказывание вывода C)
\item $\beta\vee\gamma\rightarrow(\alpha\&\beta)\vee(\alpha\&\gamma)$ (M. P. 6, 3)
\item $(\alpha\&\beta)\vee(\alpha\&\gamma)$ (M. P. 7, 5)
\end{enumerate}
И, как и раньше, применим теорему о дедукции: 
$$\vdash\alpha\&(\beta\vee\gamma)\rightarrow(\alpha\&\beta)\vee(\alpha\&\gamma)$$
\end{enumerate}

\item $\vdash\neg(\alpha\&\beta) \rightarrow \neg\alpha\vee\neg\beta$
\newline
Возможно, вас уже утомили многочисленные реверансы в сторону формального определения
доказательства, приводящие к многословным пояснениям в стиле <<здесь вставим результат
применения теоремы о дедукции к формуле>>. Давайте поэтому рассуждать не в терминах
доказательств как таковых, а в терминах возможности их получения (т.е. введем
некий более высокоуровневый язык для описаний доказательств, который просто
является сахаром над формальным определением доказательства).

Легко показать, что если $\alpha$ --- аксиома, то при любом $\Gamma$ верно $\Gamma \vdash \alpha$.
Также несложно показать, что если $\Gamma \vdash \alpha$ и $\Gamma \vdash \alpha\rightarrow\beta$, то
$\Gamma \vdash \beta$. Т.е. основные свойства теории (способность вводить схемы
аксиомы и использовать правила вывода) сохраняются вне зависимости от
содержимого контекста. Мы надеемся, что после примеров, показанных выше, в каждом из этих случаев вы
без труда воспроизведёте требуемое доказательство. 

В данной теореме необходимо два раза применить правило контрапозиции (которое будет доказано в пункте (е)).
Также, тут часто используется "временный" вынос допущений в левую часть турникета: таким образом 
доказательство получается менее громоздким и удобочитаемым. Пример такого выноса виден в шагах 3--6.
Основная идея доказательства же состоит в получении $\neg(\neg\alpha \lor \neg\beta) \to \alpha$ и 
$\neg(\neg\alpha \lor \neg\beta) \to \beta$ симметричными способами. Далее $\alpha$ и $\beta$ из 
правых частей формул объединяются в $\alpha \& \beta$ и используя контрапозицию мы получаем почти то,
что искали (необходимо еще убрать двойное отрицание в правой части импликации). Подробнее:
\begin{enumerate}[label=(\arabic*)]
\item $\vdash \neg\alpha \to (\neg\alpha \lor \neg\beta)$ (сх. акс. 6)
\item $\vdash \neg(\neg\alpha \lor \neg\beta) \to \neg\neg\alpha$ (контрапозиция, будет доказана в пункте (e))
\item $\neg(\neg\alpha \lor \neg\beta) \vdash \neg\neg\alpha$ (т. о дедукции)
\item $\neg(\neg\alpha \lor \neg\beta) \vdash \neg\neg\alpha \to \alpha$ (сх. акс. 10)
\item $\neg(\neg\alpha \lor \neg\beta) \vdash \alpha$ (M. P. 4, 3)
\item $\vdash \neg(\neg\alpha \lor \neg\beta) \to \alpha$ (т. о дедукции)
\item $\vdash \neg(\neg\alpha \lor \neg\beta) \to \beta$ (аналогично шагам 1--6)
\item $\neg(\neg\alpha \lor \neg\beta) \vdash \alpha$ (т. о дедукции)
\item $\neg(\neg\alpha \lor \neg\beta) \vdash \beta$ (т. о дедукции)
\item $\neg(\neg\alpha \lor \neg\beta) \vdash \alpha \to \beta \to \alpha \& \beta$ (сх. акс. 3)
\item $\neg(\neg\alpha \lor \neg\beta) \vdash \beta \to \alpha \& \beta$ (M. P. 10, 8)
\item $\neg(\neg\alpha \lor \neg\beta) \vdash \alpha \& \beta$ (M. P. 11, 9)
\item $\vdash \neg(\neg\alpha \lor \neg\beta) \to \alpha \& \beta$ (т. о дедукции)
\item $\vdash \neg(\alpha \& \beta) \to \neg\neg(\neg\alpha \lor \neg\beta)$ (контрапозиция, будет доказана в пункте (e))
\item $\neg(\alpha \& \beta) \vdash \neg\neg(\neg\alpha \lor \neg\beta)$ (сх. акс. 10)
\item $\neg(\alpha \& \beta) \vdash \neg\neg(\neg\alpha \lor \neg\beta) \to (\neg\alpha \lor \neg\beta)$ (сх. акс. 10)
\item $\neg(\alpha \& \beta) \vdash (\neg\alpha \lor \neg\beta)$ (M. P. 16, 15)
\item $\vdash \neg(\alpha \& \beta) \to (\neg\alpha \lor \neg\beta)$ (т. о дедукции)
\end{enumerate}

\newpage
\item $\vdash(\alpha\rightarrow\beta)\rightarrow(\neg\beta\rightarrow\neg\alpha)$
\newline
Использование теоремы о дедукции дважды в самом начале (внос $\alpha \to \beta$ и $\neg\beta$ слева от турникета) сильно облегчает дальнейшее доказательство. С помощью девятой схемы аксиом получаем $(\alpha \to \beta) \to (\alpha \to \neg\beta) \to \neg\alpha$. Заметим, что $\alpha \to \beta$ уже состоит слева от турникета, а значит может быть получена справа от турникета как гипотеза. Также, $\alpha \to \neg\beta$ может быть выведена с помощью первой схемы аксиом и гипотезы $\neg\beta$. Подробнее:
\begin{enumerate}[label=(\arabic*)]
\item $\alpha \to \beta, \neg\beta \vdash (\alpha \to \beta) \to (\alpha \to \neg\beta) \to \neg\alpha$ (сх. акс. 9)
\item $\alpha \to \beta, \neg\beta \vdash \alpha \to \beta$ (гипотеза)
\item $\alpha \to \beta, \neg\beta \vdash (\alpha \to \neg\beta) \to \neg\alpha$ (M. P. 1, 2)
\item $\alpha \to \beta, \neg\beta \vdash \neg\beta \to \alpha \to \neg\beta$ (сх. акс. 1)
\item $\alpha \to \beta, \neg\beta \vdash \neg\beta$ (гипотеза)
\item $\alpha \to \beta, \neg\beta \vdash \alpha \to \neg\beta$ (M. P. 4, 5)
\item $\alpha \to \beta, \neg\beta \vdash \neg\alpha$ (M. P. 3, 6)
\item $\vdash (\alpha \to \beta) \to (\neg\beta \to \neg\alpha)$ (т. о дедукции)
\end{enumerate}
\end{enumerate}

\end{document}
