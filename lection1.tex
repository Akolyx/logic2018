\section{Общие замечания}

\subsection{О тексте}

Данный текст представляет из себя краткий конспект лекций по курсу
<<Математическая логика>>, рассказанных студентам ИТМО (группы M3234-M3239)
в 2017-2018 учебном году.

\subsection{Общие определения и обозначения}

Прежде чем приступить к изложению содержательного материала, введём несколько
базовых определений и обозначений, которые должны быть уже знакомы читателю.

\begin{enumerate}
\item \emph{Множество всех подмножеств} обозначим как $\mathcal{P}$:
$\mathcal{P}(X) = \{ C \mid C \subseteq X \}$

\item \emph{Упорядоченную пару} каких-либо значений $a$ и $b$ 
будем обозначать как $\langle a, b \rangle$

\item Пусть дано некоторое частично-упорядоченное отношением $\sqsubseteq$ множество $S$.
\emph{Наименьшим} (\emph{наибольшим}) элементом множества назовём такой элемент $t \in S$, 
что для любого $s \in S$ выполнено $t \sqsubseteq s$ ($s \sqsubseteq t$).

\item Пусть дано некоторое частично-упорядоченное отношением $\sqsubseteq$ множество $S$.
\emph{Минимальным} (\emph{максимальным}) элементом множества назовём такой элемент $t \in S$, 
что не существует большего (меньшего) $s \in S$. Иными словами, нет такого $s$, что
$s \sqsubseteq t$ ($t \sqsubseteq s$) и $s \ne t$.

Заметим, что наименьшее (наибольшее) значение всегда единственное, а минимальных
(максимальных) значений может быть много.

\end{enumerate}

\section{Общая топология}

Мы начинаем курс немного издалека: от некоторых базовых тем общей топологии.
С одной стороны, эти знания пригодятся нам дальше в курсе, с другой --- есть 
надежда, что они настроят слушателей курса на правильный лад.

\begin{definition} \emph{Топологическим пространством} мы назовём упорядоченную
пару множеств $\langle X, \Omega \rangle$, где $\Omega \subseteq {\mathcal{P}}(X)$,
отвечающую следующим трём свойствам:

\begin{enumerate}
\item Какое бы ни было семейство множеств $\{A_\alpha\}$, где $A_\alpha \in \Omega$, выполнено
$\cup_\alpha\{A_\alpha\} \in \Omega$
\item Какое бы ни было конечное семейство множеств $\{A_1, \dots, A_n\}$, где $A_i \in \Omega$,
выполнено $A_1 \cap A_2 \cap \dots \cap A_n \in \Omega$
\item $\varnothing \in \Omega$, $X \in \Omega$
\end{enumerate}
\end{definition}

\begin{definition} Пусть дано топологическое пространство $\langle X, \Omega \rangle$.
Тогда любое множество $A \in \Omega$ назовём \emph{открытым}. Если же $X \setminus A \in \Omega$, 
то такое множество назовём \emph{замкнутым}.
\end{definition}

\begin{theorem} Следующие объекты являются топологическими пространствами:
\begin{enumerate}
\item Топология стрелки: $\langle \mathbb{R}, \{(x,+\infty) | x \in \{-\infty\}\cup\mathbb{R}\cup\{+\infty\}\} \rangle$
\item Дискретная топология на множестве $X$: $\langle X, {\mathcal{P}}(X) \rangle$
\item Топология Зарисского на множестве $X$: $\langle X, \{ A \in {\mathcal{P}}(X) | (X \setminus A) ~\textrm{--- конечно}~\} \rangle$
\end{enumerate}
\end{theorem}

\begin{definition}
\emph{Внутренностью} множества $X$ (обозначается как $\mathrm{int} X$) мы назовём наибольшее по включению окрытое подмножество $X$.
\emph{Замыканием} множества $X$ (обозначается как $\mathrm{cl} X$) мы назовём наименьшее по включению замкнутое надмножество $X$.
\end{definition}

\begin{definition} \emph{Базой} топологического пространства $\langle X, \Omega \rangle$ назовём
любое такое семейство множеств $\mathcal{B}$, что каждое открытое множество представляется
объединением некоторого подмножества $\mathcal{B}$. Или, в формальной записи,
$\Omega = \{\cup S | S \subseteq B\}$.
Также будем говорить, что данная база $\mathcal{B}$ \emph{задаёт} топологическое 
пространство $\langle X, \Omega \rangle$.
\end{definition}

\begin{theorem} Классическая топология Евклидова пространства $\mathbb{R}$:
    Множество $$\mathcal{B} = \{(a,b)|a,b\in \mathbb{R}\}$$ является базой Евклидова пространства.
\end{theorem}

\begin{definition} Топологическое пространство $\langle X, \Omega \rangle$ назовём 
связным, если единственные одновременно открытые и замкнутые множества в нём --- $\varnothing$ и $X$.
\end{definition}

\begin{theorem} Топологическое пространство $\langle X, \Omega \rangle$ связно тогда и только тогда, 
когда в нём нет двух непустых открытых множеств $A$ и $B$, что $A \cup B = X$ и $A \cap B = \varnothing$.
\end{theorem}

\begin{definition} Назовём частичным порядком $(\sqsubseteq)$ на множестве $X$ любое 
рефлексивное, транзитивное и антисимметричное отношение на нём.
\end{definition}

\begin{definition} Рассмотрим множество $X$ с заданным на нём частичным порядком $\sqsubseteq$.
Рассмотрим множество $\mathcal{B}_\sqsubseteq = \{ \{ t \in X | x \sqsubseteq t \}| x \in X\}$.
Тогда топологическое пространство $X_\sqsubseteq$, задаваемое базой топологии $\mathcal{B}_\sqsubseteq$,
мы назовём \emph{топологией частичного порядка} $(\sqsubseteq)$ на $X$.
\end{definition}

\begin{theorem} При любом выборе $X$ и $(\sqsubseteq)$ $X_\sqsubseteq$ является топологическим пространством.
\end{theorem}

\begin{definition} Пусть задано топологическое пространство $\langle X, \Omega \rangle$, и пусть
задано множество $A \subseteq X$. Тогда рассмотрим $\Omega_A = \{ S \cap A | S \in \Omega \}$.
Будем называть топологическое пространство $\langle A, \Omega_A \rangle$ пространством с топологией,
индуцированной пространством $\langle X, \Omega \rangle$.
\end{definition}

\begin{theorem} При любом выборе топологического пространства $\langle X, \Omega \rangle$ и 
$A$ (подмножества $X$) пространство с индуцированной топологией 
$\langle A, \Omega_A \rangle$ является топологическим пространством.
\end{theorem}

\begin{theorem} Пусть задано топологическое пространство $\langle X, \Omega \rangle$, и пусть
$A \subseteq X$. Тогда множество $A$ называется связным, если оно связно как пространство
с индуцированной пространством $\langle X, \Omega \rangle$ топологией.
\end{theorem}

\begin{theorem} Рассмотрим ациклический граф $G$ с множеством вершин $V$. Построим по нему
отношение: положим, что $x \sqsubseteq y$, если имеется путь из $x$ в $y$.
Тогда граф слабо связен тогда и только тогда, когда связно соответствующее топологическое 
пространство частичного порядка. 
\end{theorem}
