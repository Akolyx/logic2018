\documentclass[11pt,a4paper,oneside]{article}
\usepackage[utf8]{inputenc}
\usepackage[english,russian]{babel}
\usepackage{amssymb}
%\usepackage{amsmath}
%\usepackage{mathabx}
\usepackage{stmaryrd}
\usepackage[left=2cm,right=2cm,top=2cm,bottom=2cm,bindingoffset=0cm]{geometry}
\usepackage{bnf}
\newcommand{\lit}[1]{\mbox{`\texttt{#1}'}}
\newcommand{\ntm}[1]{<\mbox{#1}>}
\begin{document}


\begin{center}
\begin{Large}{\bfseries Домашние задания по курсу <<Математическая логика>>}\end{Large}\\
\vspace{1mm}
\begin{small} ИТМО, группы M3234..M3239\end{small}\\
\small Весна 2018 г.
\end{center}

%\renewcommand{\abstractname}{Общие замечания}
%\begin{abstract}
\subsection*{Общие замечания}
Для всех программ кодировка входных и выходных файлов должна быть UTF8. Задания 
подаются в систему Яндекс.контест, подробные описания --- по ссылке из README.md.
Для компиляции решения требуется использования мэйкфайлов, краткое описание
принципов построения мэйкфайлов находится в файле make.pdf из данного репозитория.
%\end{abstract}

\subsection*{Задача 0. Разбор выражения}
{\it Стоимость: 0 баллов, решение на Ocaml или Haskell: 0 баллов}\vspace{2mm}\\

Данная задача разобрана, решения её приведены (см. README.md), однако, мы крайне рекомендуем
написать своё её решение по двум причинам: (а) разбор высказываний можно будет переиспользовать 
в других задачах; (б) можно протестировать среду исполнения на Яндексе.

На вход программе (в файле \texttt{input.txt}) подаётся выражение в следующей грамматике:
\vspace{-1mm}
\begin{bnf}\begin{eqnarray*}
\ntm{файл} &::=& \ntm{выражение}\\
\ntm{выражение} &::=& \ntm{дизъюнкция} | \ntm{дизъюнкция} \lit{->} \ntm{выражение}\\
\ntm{дизъюнкция} &::=& \ntm{конъюнкция} | \ntm{дизъюнкция} \lit{|} \ntm{конъюнкция}\\
\ntm{конъюнкция} &::=& \ntm{отрицание} | \ntm{конъюнкция} \lit{\&} \ntm{отрицание}\\
\ntm{отрицание} &::=& (\lit{A} \dots \lit{Z}) \{\lit{A}\dots\lit{Z}|\lit{0}\dots\lit{9}\}^* | \lit{!} \ntm{отрицание} | \lit{(} \ntm{выражение} \lit{)}
\end{eqnarray*}\end{bnf}%

Пробелы, символы табуляции и переноса строки должны игнорироваться.
Символ `\texttt{|}' имеет ASCII-код $124_{10}$.

Написать программу, разбирающую выражение и строящую его дерево разбора, и
выводящую полученное дерево в файл \texttt{output.txt} в следующей грамматике.
\vspace{-1mm}
\begin{bnf}\begin{eqnarray*}
\ntm{файл} &::=& \ntm{вершина}\\
\ntm{вершина} &::=& \lit{(}\ntm{знак}\lit{,}\ntm{вершина}\lit{,}\ntm{вершина}\lit{)}\\
                &|& \lit{(!}\ntm{вершина}\lit{)}\\
                &|& (\lit{A} \dots \lit{Z}) \{\lit{A}\dots\lit{Z}|\lit{0}\dots\lit{9}\}^*\\
\ntm{знак} &::=& \lit{\&} | \lit{|} | \lit{->}
\end{eqnarray*}\end{bnf}%

\subsubsection*{Пример входного файла:}
\begin{verbatim}
P->!QQ->!R10&S|!T&U&V
\end{verbatim}

\subsubsection*{Выходной файл для данного входного файла:}
\begin{verbatim}
(->,P,(->,(!QQ),(|,(&,(!R10),S),(&,(&,(!T),U),V))))
\end{verbatim}

\subsection*{Задача 1. Проверка вывода}
{\it \textbf{ДЕДЛАЙН:} 23:59, 8 апреля }\vspace{2mm}\\
{\it Стоимость: 7 баллов, решение на Ocaml или Haskell: 9 баллов }\vspace{2mm}\\

Написать программу, проверяющую вывод $\gamma_1, \dots \gamma_n \vdash \alpha$ в исчислении 
высказываний на корректность. Входной файл соответствует следующей грамматике, нетерминал
\begin{bnf}$\ntm{выражение}$\end{bnf} определён в грамматике из задачи 0:

\begin{bnf}\begin{eqnarray*}
\ntm{файл} &::=& \ntm{заголовок} \lit{\textbackslash{}n} \{ \ntm{выражение} \lit{\textbackslash{}n}\}^*\\
\ntm{заголовок} &::=& \left[\ntm{выражение} \left\{ \lit{,}\ntm{выражение}\right\}^*\right] \lit{|-} \ntm{выражение}
\end{eqnarray*}\end{bnf}%

В первой строке входного файла (заголовок) перечислены предположения $\gamma_i$ (этот список может быть пустым) и 
доказываемое утверждение $\alpha$. В последующих строках указаны формулы, составляющие вывод формулы $\alpha$.
Пробелы, символы табуляции и возврата каретки (ASCII-код $13_{10}$) должны игнорироваться. 
Символ `\texttt{|}' имеет ASCII-код $124_{10}$.

Результатом работы программы должен быть файл с проаннотированным текстом доказательства,
где каждая строка ---
соответствующая строка из вывода, расширенная в соответствии с грамматикой:
\begin{bnf}\begin{eqnarray*}
\ntm{строка} &::=& \lit{(} \ntm{номер} \lit{) } \ntm{выражение} \lit{ (} \ntm{аннотация} \lit{)}\\
\ntm{аннотация} &::=& \lit{Сх. акс. } \ntm{номер} \\
        &|& \lit{Предп. } \ntm{номер}\\
                &|& \lit{M.P. } \ntm{номер}\lit{, }\ntm{номер}\\
                &|& \lit{Не доказано}\\
\ntm{номер} &::=& \{\lit{0}\dots\lit{9}\}^+
\end{eqnarray*}\end{bnf}%

Выражение не должно содержать пробелов, номер от выражения и выражение от аннотации должны
отделяться одним пробелом. Выражения в доказательстве должны нумероваться подряд
натуральными числами с 1. Если выражение $\delta_n$ получено из 
$\delta_i$ и $\delta_j$, где $\delta_j \equiv \delta_i\rightarrow\delta_n$
путём применения правила Modus Ponens, то аннотация должна выглядеть как 
\lit{M.P. $i$, $j$}, обратный порядок номеров не допускается.

\subsubsection*{Ограничения}
Количество строк в файле не превосходит $52000$. 
\newline
Размер файла не превосходит $10$ мегабайт.

\subsubsection*{Пример 1:}
\begin{minipage}[t]{.5\textwidth}
\subsubsection*{Входной файл:}
\begin{verbatim}
A,B|-A&B
A
B
A->B->A&B
B->A&B
A&B
\end{verbatim}
\end{minipage}
\begin{minipage}[t]{.5\textwidth}
\subsubsection*{Выходной файл:}
\begin{verbatim}
(1) A (Предп. 1)
(2) B (Предп. 2)
(3) (A->(B->(A&B))) (Сх. акс. 3)
(4) (B->(A&B)) (M.P. 3, 1)
(5) (A&B) (M.P. 4, 2)
\end{verbatim}
\end{minipage}

\subsubsection*{Пример 2:}
\begin{minipage}[t]{.5\textwidth}
\subsubsection*{Входной файл:}
\begin{verbatim}
A,B|-A&B
A
B
(A->(B->(A&B)))
(B->(A&B))
(A->A)
(A&B)
\end{verbatim}
\end{minipage}
\begin{minipage}[t]{.5\textwidth}
\subsubsection*{Выходной файл:}
\begin{verbatim}
(1) A (Предп. 1)
(2) B (Предп. 2)
(3) (A->(B->(A&B))) (Сх. акс. 3)
(4) (B->(A&B)) (M.P. 3, 1)
(5) (A->A) (Не доказано)
(6) (A&B) (M.P. 4, 2)
\end{verbatim}
\end{minipage}

\subsubsection*{Пример 3:}
\begin{minipage}[t]{.5\textwidth}
\subsubsection*{Входной файл:}
\begin{verbatim}
|-A->A
(A->A->A)->(A->(A->A)->A)->(A->A)
(A->A->A)
(A->(A->A)->A)
(A->(A->A)->A)->(A->A)
A->A
\end{verbatim}
\end{minipage}
\begin{minipage}[t]{.5\textwidth}
\subsubsection*{Выходной файл:}
\begin{verbatim}
(1) (A->A->A)->(A->(A->A)->A)->A->A (Сх. акс. 2)
(2) A->A->A (Сх. акс. 1)
(3) A->(A->A)->A (Сх. акс. 1)
(4) (A->(A->A)->A)->A->A (M.P. 1, 2)
(5) A->A (M.P. 4, 3)
\end{verbatim}
\end{minipage}

\subsubsection*{Пример 4:}
\begin{minipage}[t]{.5\textwidth}
\subsubsection*{Входной файл:}
\begin{verbatim}
|-B
A->B
A
B
\end{verbatim}
\end{minipage}
\begin{minipage}[t]{.5\textwidth}
\subsubsection*{Выходной файл:}
\begin{verbatim}
(1) (A->B) (Не доказано)
(2) A (Не доказано)
(3) B (M.P. 1, 2)
\end{verbatim}
\end{minipage}

\subsection*{Задача 2. Теорема о дедукции}
{\it \textbf{ДЕДЛАЙН:} 23:59, 15 апреля }\vspace{2mm}\\
{\it Стоимость: 4 балла, решение на Ocaml или Haskell: 6 баллов }\vspace{2mm}\\
Написать программу, преобразующую вывод $\Gamma, \alpha \vdash \beta$ в вывод
$\Gamma \vdash \alpha \rightarrow \beta$.
Входной файл удовлетворяет грамматике из предыдущего задания,
в заголовке обязательно должно присутствовать как минимум одно предположение.

Результатом работы программы должен быть текст, содержащий преобразованный вывод.
Формат выходного файла совпадает с форматом входного файла.
Вы можете предполагать, что входной файл содержит корректный вывод требуемой формулы.

\subsubsection*{Ограничения}
Небольшие.

\subsubsection*{Пример 1:}
\begin{minipage}[t]{.5\textwidth}
\subsubsection*{Входной файл:}
\begin{verbatim}
A,A|-A
A
\end{verbatim}
\end{minipage}
\begin{minipage}[t]{.5\textwidth}
\subsubsection*{Выходной файл:}
\begin{verbatim}
A|-A->A
A->A->A
A
A->A
\end{verbatim}
\end{minipage}

\subsubsection*{Пример 2:}
\begin{minipage}[t]{.5\textwidth}
\subsubsection*{Входной файл:}
\begin{verbatim}
A|-B->A
A->B->A
A
B->A
\end{verbatim}
\end{minipage}
\begin{minipage}[t]{.5\textwidth}
\subsubsection*{Выходной файл:}
\begin{verbatim}
|-A->B->A
A->B->A
\end{verbatim}
\end{minipage}


\subsection*{Задача 3. Теорема о полноте исчисления высказываний}
{\it \textbf{ДЕДЛАЙН:} 23:59, 2 мая }\vspace{2mm}\\
{\it Стоимость: 10 баллов, решение на Ocaml или Haskell: 13 баллов }\vspace{2mm}\\

Будем называть формулу классического исчисления высказываний $\phi$ 
\emph{логическим следствием} формул $\gamma_1, \dots, \gamma_n$ (и записывать это как 
$\gamma_1, \dots, \gamma_n \models \phi$), если для любой оценки пропозициональных 
переменных $M$, такой, что $\llbracket\gamma_k\rrbracket_M = \mbox{И}$, выполнено
$\llbracket\phi\rrbracket_M = \mbox{И}$. Иными словами, формула $\phi$ истинна
всегда, когда истинны все $\gamma_k$.

Написать программу, проверяющую $\gamma_1, \dots, \gamma_n \models \phi$ и строящую
доказательство $\gamma_1, \dots, \gamma_n \vdash \phi$ в случае успешной проверки,
либо строящую контрпример в случае неуспеха.

Входной файл состоит из единственной строки:
\begin{bnf}\begin{eqnarray*}
[\{\ntm{выражение}\lit{,}\}^*\ntm{выражение}]\lit{|=}\ntm{выражение}
\end{eqnarray*}\end{bnf}%
Выходной файл должен либо содержать доказательство высказывания (в формате входного файла из 
первого задания), либо содержать фразу, удовлетворяющую грамматике:
\begin{bnf}\begin{eqnarray*}
\ntm{строка} &::=& \lit{Высказывание ложно при } ~\ntm{назначение} \{\lit{,} \ntm{назначение} \}^*\\
\ntm{назначение} &::=& \ntm{переменная} \lit{=} (\lit{И}|\lit{Л})
\end{eqnarray*}\end{bnf}%

\subsubsection*{Ограничения}
Количество связок не превосходит 12 (например в выражении $A \wedge B \vdash A \vee B$ --- 2 связки). \\
Количество различных переменных не превосходит 5.

\subsubsection*{Пример 1:}
\begin{minipage}[t]{.5\textwidth}
\subsubsection*{Входной файл:}
\begin{verbatim}
|=!A&!B
\end{verbatim}
\end{minipage}
\begin{minipage}[t]{.5\textwidth}
\subsubsection*{Выходной файл:}
\begin{verbatim}
Высказывание ложно при A=И, B=Л
\end{verbatim}
\end{minipage}

\subsubsection*{Пример 2:}
\begin{minipage}[t]{.5\textwidth}
\subsubsection*{Входной файл:}
\begin{verbatim}
B,W|=A->B
\end{verbatim}
\end{minipage}
\begin{minipage}[t]{.5\textwidth}
\subsubsection*{Выходной файл:}
\begin{verbatim}
B,W|-A->B
B->A->B
B
A->B
\end{verbatim}
\end{minipage}

\subsection*{Задача 4. Решётки}
{\it \textbf{ДЕДЛАЙН:} 23:59, 20 мая }\vspace{2mm}\\
{\it Стоимость: 6 баллов, решение на Ocaml или Haskell: 8 баллов }\vspace{2mm}\\

По заданному на вход вашей программе графу требуется установить, задаёт ли
его рефлексивное и транзитивное замыкание
решётку, а также, является ли она дистрибутивной, импликативной решёткой,
булевой алгеброй. Гарантируется, что рефлексивное и транзитивное замыкание
графа задаёт частичный порядок.

Вершины графа мы предполагаем занумерованными числами от $1$ до $v$.
Входной файл в первой строке содержит число вершин $v$, после чего идёт
ещё $v$ строк, по строке для каждой вершины: вершине $i$ соответствует строка
номер $i+1$ входного файла. В каждой такой строке через пробел перечислены все такие 
вершины $i_k$, что $i \sqsubseteq i_k$. Гарантируется, что все эти строки
содержат хотя бы по одной вершине.

Выходной файл должен соответствовать следующей грамматике:

\begin{bnf}\begin{eqnarray*}
\ntm{выходной файл} &::=& \lit{Операция '+' не определена: } \ntm{вершина}\lit{+}\ntm{вершина} \\
                    &|& \lit{Операция '*' не определена: }\ntm{вершина}\lit{*}\ntm{вершина} \\
                    &|& \lit{Нарушается дистрибутивность: }\ntm{вершина}\lit{*(}\ntm{вершина}\lit{+}\ntm{вершина}\lit{)} \\
                    &|& \lit{Операция '->' не определена: }\ntm{вершина}\lit{->}\ntm{вершина} \\
                    &|& \lit{Не булева алгебра: }\ntm{вершина}\lit{+\textasciitilde}\ntm{вершина}\\
                    &|& \lit{Булева алгебра}
\end{eqnarray*}\end{bnf}

Вам следует находить самое слабое свойство (например, если граф не является решёткой, то
указание на нарушение дистрибутивности будет ошибкой). \emph{Для данной задачи} будем 
считать, что чем ниже свойство указано в данной грамматике, тем оно сильнее.

\subsubsection*{Ограничения}
Количество вершин в графе не превосходит $100$.

\subsubsection*{Пример 1:}
\begin{minipage}[t]{.5\textwidth}
\subsubsection*{Входной файл:}
\begin{verbatim}
5
2 3 4
5
5
5
5
\end{verbatim}
\end{minipage}
\begin{minipage}[t]{.5\textwidth}
\subsubsection*{Выходной файл:}
\begin{verbatim}
Нарушается дистрибутивность: 2*(3+4)
\end{verbatim}
\end{minipage}


\subsection*{Задача 5. Опровержение формулы ИИВ}
{\it \textbf{ДЕДЛАЙН:} 23:59, 20 июня }\vspace{2mm}\\
{\it Стоимость: 12 баллов, решение на Ocaml или Haskell: 15 баллов }\vspace{2mm}\\

На вход программе дана формула ИИВ. Требуется построить либо
алгебру Гейтинга, опровергающую формулу, либо указать, что формула общезначима.

Формат входного файла --- аналогично задаче 3.
Формат выходного файла:

\begin{bnf}\begin{eqnarray*}
\ntm{выходной файл} &::=& \ntm{задание графа}\lit{\textbackslash n}\ntm{переменные}\\
                    &|& \lit{Формула общезначима}\\
\ntm{переменные}    &::=& \ntm{имя переменной}\lit{=}\ntm{номер вершины} [ \lit{,}\ntm{переменные} ]
\end{eqnarray*}\end{bnf}

Формат задания графа соответствует формату из входного файла в задаче 4.

\begin{verbatim}
Формула общезначима
\end{verbatim}

\subsubsection*{Ограничения}
Входная формула имеет не более трёх переменных.

\subsubsection*{Пример 1:}
\begin{minipage}[t]{.5\textwidth}
\subsubsection*{Входной файл:}
\begin{verbatim}
A|!A
\end{verbatim}
\end{minipage}
\begin{minipage}[t]{.5\textwidth}
\subsubsection*{Выходной файл:}
\begin{verbatim}
3
1
1 2
1 2 3
A=2
\end{verbatim}
\end{minipage}

\subsubsection*{Пример 2:}
\begin{minipage}[t]{.5\textwidth}
\subsubsection*{Входной файл:}
\begin{verbatim}
A->A
\end{verbatim}
\end{minipage}
\begin{minipage}[t]{.5\textwidth}
\subsubsection*{Выходной файл:}
\begin{verbatim}
Формула общезначима
\end{verbatim}
\end{minipage}

\subsection*{Задача 6. Построение алгебры Гейтинга по модели Крипке}
{\it \textbf{ДЕДЛАЙН:} 23:59, 20 июня }\vspace{2mm}\\
{\it Стоимость: 6 баллов, решение на Ocaml или Haskell: 8 баллов }\vspace{2mm}\\

На вход программе задаётся формула ИИВ и модель Крипке, опровергающая данную формулу.
Требуется построить по ней алгебру Гейтинга, также опровергающую формулу.

Модель Крипке задаётся в следующем формате:

\begin{bnf}\begin{eqnarray*}
\ntm{входной файл} &::=& \ntm{формула}\lit{\textbackslash{}n}
  \{\ntm{отступ}\lit{*}[\lit{ }\ntm{переменные}]\lit{\textbackslash{}n}\}^*\\
\ntm{переменные} &::=& \ntm{имя}[\lit{,}\ntm{переменные}]\\
\end{eqnarray*}\end{bnf}%
В данной модели миры образуют лес. 
Каждая строка входного файла описывает некоторый мир, 
указанные после звёздочки переменные вынуждены в соответствующем мире.

Отступ перед звёздочкой содержит только пробелы. Их количество указывает на 
вложенность соотвествующего миру узла в дереве --- подобно тому, как это делается
в Питоне. Однако, в отличие от Питона, отступ может возрастать только на 1.
А именно, пусть каждый мир $W_i$ указан в строке $i$ и имеет отступ $I_i$, тогда
мир $W_j$ --- непосредственный потомок $W_i$, если выполнены все следующие условия: 
\begin{enumerate}                                                  
\item описание мира $W_j$ идёт ниже по тексту: $j > i$;
\item его отступ ровно на 1 больше: $I_j = I_i+1$
\item между строками $i$ и $j$ нет описаний с меньшим отступом: нет $k (i < k < j)$,
что $I_i \le I_k$.
\end{enumerate}

Гарантируется, что входной файл корректен (соответствует правилам, указанным выше). 
Однако, не гарантируется, что заданное на вход множество миров с указанием вынужденных
в них переменных действительно задаёт модель Крипке и что формула действительно ею
опровергается.

Формат выходного файла аналогичен формату из задачи 5:

\begin{bnf}\begin{eqnarray*}
\ntm{выходной файл} &::=& \ntm{задание графа}\lit{\textbackslash n}\ntm{переменные}\\
                    &|& \lit{Не модель Крипке}\\
                    &|& \lit{Не опровергает формулу}\\
\ntm{переменные}    &::=& \ntm{имя переменной}\lit{=}\ntm{номер вершины} [ \lit{,}\ntm{переменные} ]
\end{eqnarray*}\end{bnf}

\subsubsection*{Ограничения}
Входной файл содержит описание не более 4 миров и не более 4 переменных.

\subsubsection*{Пример 1:}
\begin{minipage}[t]{.5\textwidth}
\subsubsection*{Входной файл:}
\begin{verbatim}
A|!A
*
 * A
\end{verbatim}
\end{minipage}
\begin{minipage}[t]{.5\textwidth}
\subsubsection*{Выходной файл:}
\begin{verbatim}
3
1
1 2
1 2 3
A=2
\end{verbatim}
\end{minipage}

\subsubsection*{Пример 2:}
\begin{minipage}[t]{.5\textwidth}
\subsubsection*{Входной файл:}
\begin{verbatim}
A|B
* A,B
  * A
* B
\end{verbatim}
\end{minipage}
\begin{minipage}[t]{.5\textwidth}
\subsubsection*{Выходной файл:}
\begin{verbatim}
Не модель Крипке
\end{verbatim}
\end{minipage}

\end{document}
