\subsection{Булева алгебра и Топологическая интерпретация интуиционистского исчисления высказываний}

Мы построим две параллельные интерпретации для классической и интуиционистской логики.

\begin{definition}
\end{definition}

Пусть дано некоторое исчисление высказываний, для которого нам нужно
построить модель --- предложить способ оценки истинности выражений.
Начинаем мы с множества истинностных значений.
Возьмем в качестве этого множества все открытые множества некоторого
заранее выбранного топологического пространства.
Определим оценку для связок интуиционистского исчисления высказываний следующим образом:

\begin{tabular}{l}\\
$\llbracket A \& B \rrbracket = \llbracket A \rrbracket \cap \llbracket B \rrbracket$\\
$\llbracket A \vee B \rrbracket = \llbracket A \rrbracket \cup \llbracket B \rrbracket$\\
$\llbracket A \rightarrow B \rrbracket = (c\llbracket A \rrbracket \cup \llbracket B \rrbracket)^\circ$\\
$\llbracket \neg A \rrbracket = (c \llbracket A \rrbracket)^\circ$
\end{tabular}

Будем считать, что формула истинна в данной модели, если её значение оказалось равно
всему пространству. 

Например, возьмем в качестве пространства $\mathbb{R}$, и вычислим значение формулы $A \vee \neg A$ 
при $A$ равном $(0,1)$: $\llbracket A \vee \neg A \rrbracket = (0,1) \cup \llbracket \neg A \rrbracket = 
(0,1) \cup (c(0,1))^\circ = (0,1) \cup ((-\infty,0)\cup(1,\infty)) = (-\infty,0)\cup(0,1)\cup(1,\infty)$.
Нетрудно видеть, что данная формула оказалась не общезначимой в данной интерпретации.

